Seismic processing technologies offer a range of options for addressing a given problem, typically presenting a choice between low-cost, low-resolution solutions and high-cost, high-resolution solutions. With advancements in deep learning, users now have a third option: mapping low-quality data to high-quality results. In this thesis, we have explored and optimised three models, namely GAN, SCRN and SCR-GAN, for the MPFI-to-TDRI seismic image translation purpose. These models effectively produced high-quality TDRI images. Based on our evaluation metrics and slice reconstruction performance, SCRN stands out as the most effective model and its de-noising capabilities, potentially surpassing TDRI itself. Once trained, the SCRN model could produce TDRI quality result through the low-to-high quality translation workflow, with computational costs comparable to MPFI processing. 
\\\\
\textit{Is there a way do high-quality seismic interpolation fast and cheap?} Yes, there is.
\\\\
While the performance is satisfactory, our current model is trained exclusively on Gulf of Mexico data, limiting its generalisation. To enhance robustness, future work should involve expanding the training dataset with diverse locations and common receiver gathers, as well as augmenting data through frequency filtering. This low-to-high quality image translation workflow can be extended to other seismic applications, such as deconvolution, by using pairs of low-cost and high-cost processing results.