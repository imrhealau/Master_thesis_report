%In signal processing, we follow the Nyquist frequency in sampling signals to avoid aliasing. However, seismic data are typically irregularly and sparsely collected in spatial sense due to finite surveying and time resources. Before pre-processing the dataset, we require an interpolation method that prioritizes speed and affordability while achiev-ing good accuracy. SLB inhouse interpolation methods include either matching pursuit Fourier interpolation (MPFI) or time domain Radon interpolation (TDRI) depicted in Figure 1. MPFI interpolates traces by matching data iterative-ly in the frequency domain. It is low in computational and time cost, but has limited robustness to heavily spatial aliased datasets (Schonewille et. al, 2013). TDRI performs interpolation in the Radon domain. It is more robust to aliasing and steeply dipping events, with overall better performance than MPFI, yet significantly more computation-ally expensive and time consuming \cite{schonewille2014comparison}.  
% \cite{oliveira2018interpolating}

Since \citeA{krizhevsky2012imagenet} introduced the first deep Convolutional Neural Networks (CNNs) for image classification, deep learning methodologies, especially those employing CNNs, have increasingly dominated computer vision tasks. \citeA{goodfellow2014generative} established the framework of Generative Adversarial Networks (GANs), comprising a generator and a discriminator that are trained concurrently through adversarial processes, with the generator producing images and the discriminator evaluating their authenticity. CNNs are integral parts of GAN architectures for feature extractions. \citeA{isola2017image} examined conditional adversarial networks (cGANs) as a versatile solution for image-to-image translation challenges, while \citeA{ledig2017photo} specifically applied GANs to enhance image resolution from low to high. The introduction of Transformers during the same period introduced a new class of deep learning architectures. Unlike CNNs, which capture local features, Transformers employ a self-attention mechanism that enables the modelling of long-range dependencies \cite{vaswani2017attention}. \citeA{dosovitskiy2020image} defined Vision Transformers (ViTs) tailored for computer vision applications, and \citeA{liu2021swin} refined this with the Swin Transformer, a shift-window variant. Both architectures demonstrate capabilities in image processing tasks, including translation. Leveraging the strengths of both architectures, \citeA{torbunov2023uvcgan} proposed UVCGAN, a general-purpose architecture that integrates ViTs and GANs for effective image-to-image translation.
\\\\
These deep learning image-translation methods can be applied in the seismic context to make seismic processing more affordable and accurate, in this thesis we will focus particularly on seismic interpolation. In seismic acquisition, sampling data at the Nyquist rate \(f_{Nyq} = \frac{1}{2\Delta t}\) is challenging due to high surveying and time costs, thus resulting in aliasing. To address this, seismic interpolation is used to spatially transform irregularly sampled traces to a desired grid before further processing \cite{claerbout1976fundamentals}. Seismic interpolation methods fall into two categories: wave-equation-based and signal processing methods.
\begin{itemize}
\item Wave-equation methods rely on seismic wave propagation and require a velocity model to interpret traces, allowing interpolation through complex geological structures and subsurface heterogeneities, recovering parabolic and hyperbolic events \cite{ronen1987wave}. However, these methods need accurate velocity models, which may not be available, and their straight-raypath approximation can fail for low-frequency waves \cite{stolt2002seismic}. 
\item Signal processing methods for seismic interpolation include adaptive domain transform and prediction-error filter methods, which do not require subsurface information. Studies have demonstrated their applications: F-X domain interpolation for regularly sampled data \cite{spitz1991seismic}; F-K domain interpolation and minimum weighted norm interpolation for regularly sampled data with optional missing traces \cite{naghizadeh2012seismic,liu2004minimum}; and matching pursuit Fourier interpolation (MPFI) and time domain Radon interpolation (TDRI), methods this thesis focusing on, for irregularly sampled data \cite{schonewille2013matching,schonewille2014comparison}. MPFI and TDRI excel by assuming sparsity in higher-dimensional spaces, enabling the reconstruction of highly curved and dispersed events that cannot be achieved with the assumptions of linearity and sparsity in the TX domain. TDRI achieves higher accuracy and incurs more computational cost than MPFI because TDRI's basis functions operate in the two-dimensional $\tau p$ domain, while MPFI iterates in the one-dimensional $k_x$ domain for each frequency.
\end{itemize}

\noindent Deep learning methods, such as CNNs, GANs, and transformers, do not require velocity models and provide faster reconstruction \cite{kaur2019seismic, kaur2021seismic, gao2024swin}. CNNs combine adaptive domain transforms and prediction-error filter methods without linearity or sparsity assumptions, by learning and applying transforms and filters through iterative training \cite{oliveira2018interpolating}. \citeA{gan2015dealiased}, \citeA{oliveira2018interpolating}, and \citeA{kaur2019seismic} have explored the application of GANs for interpolating missing seismic traces, demonstrating its effectiveness in reconstructing data where conventional methods might fall short. Additionally, more recent studies by \citeA{guo2023seismic} and \citeA{gao2024swin} have investigated the use of transformers for the same purpose, by using transformers to capture long-range dependencies and contextual information to accurately fill in missing seismic traces, While these methods can be applied to 3D data, they require large amounts of training data \cite{khosro2023machine}. Figure \ref{fig:methods} illustrates the various algorithms used for seismic trace interpolation and processing. 

\begin{figure}[h]
	\centering
	\includegraphics[width=0.55\textwidth]{Figure/theory/methods.png} 
	\caption{\textit{Deep learning algorithms used for seismic processing applications (taken from \citeA{khosro2023machine}).}}
	\label{fig:methods}
\end{figure}
\newpage
\noindent 
Our aim in this thesis is to combine the strengths of deep learning networks with conventional seismic pre-processing technologies to achieve optimal results to get the best of both worlds. Returning to our focus on low-to-high-quality seismic image translation, we aim to enhance seismic images quality using deep learning, specifically with GANs and ViT. The two-step workflow is as follows (illustrated in Figure \ref{fig:lhc}):

\begin{enumerate}
	\item Train the neural network with paired images: standard-resolution images obtained from low-cost processing and high-quality images from high-cost processing.
	\item Input the unseen low-cost standard-resolution datasets into the trained model to transform them into high-resolution counterparts.
\end{enumerate}

\begin{figure}[h]
	\centering
	\includegraphics[width=\textwidth]{Figure/theory/lhc.png} 
	\caption{\textit{2-Step deep learning workflow for low-to-high-quality seismic image translation.}}
	\label{fig:lhc}
\end{figure}

\noindent To demonstrate the feasibility of low-to-high-quality image translation with deep learning, we will use interpolation as an example, converting low-cost standard-resolution MPFI datasets into high-cost high-resolution TDRI solutions (see figure \ref{fig:lhc2}). Unlike previous deep learning methods for seismic interpolation that are applied on on raw traces, we are interested in transforming low-quality, conventionally interpolated traces into high-quality traces interpolated by expensive methods.

\begin{figure}[h]
	\centering
	\includegraphics[width=\textwidth]{Figure/theory/lhc2.png} 
	\caption{\textit{2-Step deep learning workflow for MPFI-to-TDRI seismic image translation.}}
	\label{fig:lhc2}
\end{figure}

\section{Scope of study}
\textit{Is there a way do high-quality seismic interpolation fast and cheap?} In this thesis, we present a novel approach for translating low-to-high-quality seismic images using deep learning, specifically employing three different models: Generative Adversarial Network (GAN), Swin Transformer Convolutional Residual Network (SCRN), and their hybrid variant, Swin Transformer Convolutional Residual Generative Adversarial Network (SCR-GAN). We have carefully selected the models' hyper-parameters through rounds of model optimisation testing to ensure optimal performance. We demonstrate the feasibility of this quality enhancement method by translating MPFI images (industrial standard quality) into their pseudo-TDRI counterparts (state-of-the-art quality) as outlined in Figure \ref{fig:lhc2}. The resemblance between the pseudo-TDRI images generated by deep learning and their ground truth will be evaluated quantitatively and qualitatively to identify the best model. This workflow is the first of its kind to produce high-resolution outputs from a low-resolution processing sequence in an end-to-end manner, effectively mitigating the computational bottleneck associated with high-resolution technology. \textbf{In recognition of its novelty and potential impact, we have filed two provisional patents for this technology during its development: one for the MPFI-TDRI translation purpose and the other for the UDD-MDD application.}

\section{Structure of thesis}
In Chapter \ref{ch:seis}, we will present both seismic processing theory and deep learning theory. The seismic processing section covers the workings of TDRI and MPFI, while the deep learning theory section introduces key concepts in deep learning and vision transformers, ensuring readers without a computer science background are well-prepared for more advanced topics later on. This is followed by descriptions and architectures of the three deep learning models used in our experiments: GAN, SCRN and SCR-GAN. Chapter \ref{ch:exp} details our experimental methodology, beginning with data preparation, followed by model training and validation, model evaluation metrics, and finally, model optimisation for each of our three models. Chapter \ref{ch:result}  presents both quantitative and qualitative results of image translation for our three models. Chapter \ref{ch:discussion} discusses the pros and cons of each model, their computation time taken, and identifies the best model for low-to-high-quality seismic image translation. Chapter \ref{ch:conclusion} summarises the overall findings of the report.